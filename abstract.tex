\begin{abstract}

%*****************************************************************************************************
% Apresentação concisa dos pontos relevantes, dando uma visao rapida e clara do conteúdo do trabalho.
%*****************************************************************************************************
% A fragmentação de arquivos tem sido o calcanhar de aquiles dos processos investigativos e perícias forenses envolvendo a recuperação de arquivos e,
% principalmente, de arquivos não mais alocados, removidos ou corrompidos. A abordagem do tema tem caminhado para um avanço nos processos de localização
% dos fragmentos de arquivos e seus diferentes tipos de técnicas, possibilitando que as ferramentas evoluam cada vez mais no processo de
% identificação e localização de arquivos fragmentados, auxiliando assim na investigação mais robusta e eficiente de evidências digitais.
The file fragmentation has been the Achilles heel of investigative processes and skills involving forensic file recovery and
mainly unallocated, removed or corrupted files. The theme has moved towards a breakthrough in the localization processes
fragments of files and their different types of techniques, enabling the tools evolve increasingly in process
identification and location of fragmented files, thus assisting in the investigation of more robust and efficient digital evidence.

\textbf{Keywords:} Fragmentation, investigation, files, techniques, evidence.

\end{abstract}