\chapter{Sistemas de Arquivos NTFS}
O sistema de arquivos NTFS foi projetado para melhor confiabilidade, seguran�a e suporte de grandes dispositivos de armazenamento de dados. Disp�e do uso de estruturas gen�ricas que servem de envolt�rio para estruturas de dados com conte�do espec�fico. Desta forma, o NTFS se torna um projeto escal�vel pois a estrutura interna de dados pode mudar in�meras vezes enquanto que a sua casca (a estrutura gen�rica) permanece constante. Um bom exemplo desse modelo � que todos os bytes\footnote{Termo bin�rio que representa 8 unidades da menor unidade de informa��o (bit) que pode ser armazenada ou transmitida \cite{WikiBytes}.} s�o alocados em arquivos no sistema \cite{ForensicAnalysis}.

\section{Conceito}
...

\section{Estrutura de Arquivos}
...

\section{Aloca��o de Arquivos}
...