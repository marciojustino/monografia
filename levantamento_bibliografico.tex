\chapter{Levantamento Bibliográfico}
Desenvolvimento do texto baseado nas referências pesquisadas, que corresponde ao corpo teórico do trabalho. Aqui, deve-se incluir a pesquisa bibliográfica dos assuntos relacionados ao tema, obtendo o corpo de conhecimentos que balizará o estudo. Os capítulos e subtítulos devem ser encadeados de forma a explicar, discutir e demonstrar o conteúdo do trabalho.
Esta etapa consiste em estruturar de maneira lógica as partes do trabalho, seus capítulos, de forma que estes estejam bem inseridos no contexto do discurso e da redação. Pode ser que, no decorrer do trabalho, esta construção lógica precise sofrer modificações ou desdobramentos, até que se chegue ao plano definitivo (Severino: 1979, p. 86). Toda a argumentação e raciocínio são construídos em cima das leituras, experiências e da vivência intelectual a respeito do problema da pesquisa.
Esta fase pressupõe o levantamento de toda a documentação existente sobre o assunto da pesquisa: livros, artigos, revistas. Conforme colocado por Severino (1979, p. 81), esta é a ”fase da heurística, ciência, técnica e arte da pesquisa de documentos”. 
É importante salientar que o trabalho, nesta etapa, pressupõe uma finalidade didática, não podendo ser uma pura criação mental do aluno. Deve se balizar em pesquisas e consultas de documentação, em livros, artigos técnicos e jornais, de forma a gerar quantidade suficiente de informação sobre o tema do trabalho. As idéias e opiniões dos alunos, da fase do levantamento bibliográfico, podem ajudar na ligação entre idéias de autores, mas não devem prevalecer sobre a pesquisa dos autores. 
É interessante que, antes de iniciar a leitura dos materiais, o aluno elabore um roteiro de trabalho (que podem ser a origem dos capítulos), ou seja, uma primeira estruturação, de forma que a leitura e a pesquisa se dêem dentro deste roteiro. Ou seja, tenha em mente quais são as colunas mestras do trabalho que demonstrarão, dos textos lidos, os elementos que devem ser retidos para o aproveitamento na composição do trabalho. Vale salientar que este roteiro é provisório, podendo ser reformulado no decorrer do trabalho.
O aluno deve proceder uma análise de possíveis materiais que podem ser interessantes para o estudo do tema e consultar seu orientador sobre livros que são imprescindíveis para a consecução do estudo. Algumas dicas válidas: tire um tempo para ir a biblioteca e analise os livros pelo seu índice. Muitas vezes, importantes teorias não aparecem no título da obra, mas sim em capítulos dos livros. Pesquisar com paciência, nesta fase, é a chave do sucesso futuro do trabalho. 
Procure analisar livros relacionados com o assunto, revistas, teses. Com certeza, esta atitude vai ajudar não apenas no estudo e detalhamento do tema em si, como também dar um foco mais interessante ao estudo ou ao problema de pesquisa. Lembre-se: pesquisar não significa perder tempo, mas sim, ganhar luz e maior proficiência sobre o assunto.
Uma última colocação acerca do levantamento bibliográfico: atualmente, os alunos tendem a pesquisar com ênfase errônea na Internet. Apesar de este ser um meio de informação altamente eficaz, certamente não é suficiente para gerar o conhecimento necessário ao desenvolvimento teórico do trabalho. Podemos comparar a Internet a um lago, que pode ser extenso, porém é raso, superficial.
Outro cuidado diz respeito à ação “copiar e colar”.  Nesta hora, o aluno deve ter amadurecimento necessário para não copiar trechos de obras, mesmo que contenha a citação da mesma, pois isso  se configura plágio. O trabalho deve ser o de ler e reformular o que foi dito pelo autor, dentro da perspectiva de seu tema. Um exemplo de reformulação: “Segundo Coda (ano), a motivação pode ser definida como...”. Perceba que, desta forma, o trecho do autor Coda não esta sendo copiado, mas comentado pelo autor da monografia.
Quando for absolutamente imprescindível copiar definições ou trechos de obras, este deve ser colocado entre aspas, de forma que fique clara a autoria de determinado trecho, e deve ser indicada a página do livro de onde foi retirada. Trechos com mais de 4 linhas devem vir em fonte menor e com margem esquerda de 4 cm.
A seguir, devem ser desenvolvidos os capítulos. Um ponto importante é fazer a ligação entre um capítulo e o seguinte, de forma que o leitor entenda completamente a sequenciação.

\chapter{Capítulo 1 - Sistemas de Arquivos}
...

\chapter{Capítulo 2 - NTFS}
...

\section{Conceito}
...

\section{Estrutura de Dados}
...

\section{Análise Típica}
...

\chapter{Capítulo 3 - File Carving}
...

\section{Conceito}
...

\section{Assinatura de Arquivo}
...

\section{Número Mágico}
...

\chapter{Capítulo 4 - File Carving Avançado}
...

\section{Conceito}
...

\section{Funcionamento}
...

\section{Fragmentação}
...

\section{Ponto de Fragmentação}
...aqui serão incluídos também algoritmos de detecção de pontos de fragmentação como listado no mind map (Sequential Hypothesis Testing)...


